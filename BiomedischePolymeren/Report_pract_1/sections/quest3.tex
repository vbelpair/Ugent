\subsection{GPC}

    A higher amount of photo-initiator at the start of the reaction will result in 
    the creation of more polymer chains, hence the average lenght (DP) of one polymer 
    will be lower than the case with lower initial initiator concentrations. Using formula 
    \refeq{eq:DPx} yields the following DP values for the different amout of initiator.

    $$\left\{\begin{matrix}
        \mathtext[1mole\%]{DP} = \frac{100}{1} = 100 \\
        \mathtext[1.5mole\%]{DP} = \frac{100}{1.5} = 66.67 \\
        \mathtext[2mole\%]{DP} = \frac{100}{2} = 50 \\
    \end{matrix}\right.$$

    The polydispersity DPI is calculated as $\frac{\mathtext[w]{M}}{\mathtext[n]{M}}$. For the 
    three different chromatograms one finds the following values

    $$\left\{\begin{matrix}
        \mathtext[GPC1]{DPI} = \frac{67016}{38480} = 1.74\\
        \mathtext[GPC2]{DPI} = \frac{58757}{30089} = 1.95\\
        \mathtext[GPC3]{DPI} = \frac{106029}{38838} = 2.73\\
    \end{matrix}\right.$$

    The DPI value is a measure on the spread of the molecular weight of the polymer. This value is, in most cases, 
    positively correlated with the DP value of the polymer since a higher average chain lenght (higher DP) results 
    in a higher variety on the polymer lenght. Hence high DPI values corresponds to high DP values. Using this correlation 
    results is the following correspondace between the GPC chromatograms and the initial initiator amounts.

    $$\left\{\begin{matrix}
        \text{chromatogram 1} \longleftrightarrow \text{2 mole\%}\\
        \text{chromatogram 2} \longleftrightarrow \text{1.5 mole\%}\\
        \text{chromatogram 3} \longleftrightarrow \text{1 mole\%}\\
    \end{matrix}\right.$$