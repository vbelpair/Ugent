\subsection{NMR spectroscopy}

Let's denote $\mathrm{I}(x)$ as the integration of a peak at a chemical shift $x$ on the H-NMR-spectra. 
It is know that vinylic protons corresponds to chemical shifts at 5.4 and 6 ppm thus the total amout of 
monomors left is found by the sum $\frac{\mathrm{I}(5.4\text{ppm})+\mathrm{I}(6\text{ppm})}{2}$ (division by two because there are two H-protons present on one vinylic group). The chemical shift corresponding to
the methoxy protons is 3.5 ppm and since the amount of methoxy protons remains the same its integration value $\frac{\mathrm{I}(3.5 \text{ppm})}{3}$  (division by three because there are three H-protons present on one methoxy group)
can be used as a reference of the total amount of monomers present at the start of the reaction. The conversion of the polymerisation reaction 
can then be calculated as

\begin{equation}
    \text{conversion of polymerisation} = 1 - \frac{3}{2}\frac{\mathrm{I}(5.4\text{ppm})+\mathrm{I}(6\text{ppm})}{\mathrm{I}(3.5 \text{ppm})}
    \label{convPoly}
\end{equation}