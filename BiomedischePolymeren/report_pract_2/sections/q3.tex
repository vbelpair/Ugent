\section{Question 3: Texturometry}

Two differences in the mechanical testing method:
\begin{itemize}
    \item[1.] Rheology measures the flow and deformation of materials under applied stress or strain, while texturometry 
    measures the mechanical properties of materials under compression, tension, or shearing forces.
    \item[2.] Rheology tests are typically performed using a rheometer, which applies precise controlled shear or tensile 
    stresses to the sample and measures the resulting deformation or stress response. Texturometry tests, on the other 
    hand, are performed using a texture analyzer, which applies a compressive, tensile, or shearing force to the sample and measures the resulting deformation or force response.
\end{itemize}

Next, two texturometry tests were performed by two researchers. Table \ref{tab:inverted-mechanical-test-results} and \ref{tab:inverted-mechanical-test-results-10wv} are results of a texture profile analysis (TPA) from researcher 1. 
Compairing the hardness, the ability to resist deformation, and springiness, the ability to recover to the original shape after compression, between the two tables indicate that higher w/v\% values results in a stiffer/harder material with a better ability to recover to its original shape after compression. 
If the DS of the material would br  

\begin{table}[H]
    \centering
    \begin{tabular}{c|ccc}
    \hline
    & Sample 1 & Sample 2 & Sample 3 \\ \hline
    Hardness1 [N] & 0.035167098 & 0.043005662 & 0.04099971 \\ 
    Hardness2 [N] & 0.030627189 & 0.032652594 & 0.040604329 \\ 
    Area1 [Nmm] & 0.003338952 & 0.002307124 & 0.003828073 \\ 
    Area2 [Nmm] & 0.001615961 & 0.000898191 & 0.002239604 \\ 
    Cohesiveness & 0.483972601 & 0.389311989 & 0.585047474 \\ 
    Springiness [mm] & 0.282010363 & 0.218791648 & 0.270347203 \\ 
    Springiness Index & 0.531212397 & 0.402065399 & 0.544778107 \\ 
    Gumminess [N] & 0.017019912 & 0.01674262 & 0.023986777 \\ 
    Chewiness [Nm] & 4.79979E-06 & 3.66315E-06 & 6.48476E-06 \\
    \end{tabular}
    \caption{TPA results - 5 w/v\%}
    \label{tab:inverted-mechanical-test-results}
\end{table}

\begin{table}[H]
    \centering
    \begin{tabular}{c|ccc}
    \hline
    & Sample 1 & Sample 2 & Sample 3 \\ \hline
    Hardness1 [N] & 0.067449574 & 0.06582288 & 0.072415185 \\ 
    Hardness2 [N] & 0.06262254 & 0.064647785 & 0.067325579 \\ 
    Area1 [Nmm] & 0.007224188 & 0.006074186 & 0.008424538 \\ 
    Area2 [Nmm] & 0.005469138 & 0.004678299 & 0.005826683 \\ 
    Cohesiveness & 0.757059224 & 0.770193572 & 0.691632352 \\ 
    Springiness [mm] & 0.465056416 & 0.38483355 & 0.420807332 \\ 
    Springiness Index & 0.889844319 & 0.767740421 & 0.767942963 \\ 
    Gumminess [N] & 0.051063322 & 0.050696359 & 0.050084685 \\ 
    Chewiness [Nm] & 2.37473E-05 & 1.95097E-05 & 2.1076E-05 \\
    \end{tabular}
    \caption{TPA results - 10 w/v\%}
    \label{tab:inverted-mechanical-test-results-10wv}
    \end{table}