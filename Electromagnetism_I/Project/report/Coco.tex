\section{A bit traveling through a lossless transmission line}
%In this section, the numerical analysis of a bit traveling through a lossless transmission line is %established ({\bf Figure CIRCUIT}). To this end, a rough analytical estimation of the simulation is %made

\subsection{General behavior of a bit traveling through a transmission line}
This section aims to predict, and thus validate, the behavior of the numerical simulation of the voltage in a LTL. A rough analytical approach is used, since it gives more insight in the physical nature of the problem.  \\

Rearranging the telegrapher's equations for a LTL yields

\begin{equation}
\frac{\partial^2\hat{v}(z, t)}{\partial z^2} - k^2\frac{\partial^2 \hat{v}(z, t)}{\partial t^2} = 0, \quad k = \frac{1}{c}= \sqrt{LC} = \mathrm{cte}.
\label{tele}
\end{equation}

This means that $\hat{v}$ satisfies the wave equation and thus can be written as a superposition of two voltage waves traveling in opposite directions with constant speed $c$. That is,

\begin{equation}
\hat{v}(z, t) = \underbrace{\hat{v}^{+}(z - ct)}_{\text{forward wave}} + \underbrace{\hat{v}^{-}(z + ct)}_{\text{backward wave}}.
\end{equation}

Realising that the voltage and current are related;

\begin{equation}
\hat{i}(z, t) = \frac{1}{R_c}(\hat{v}^{+}(z, t) - \hat{v}^{-}(z, t)),
\end{equation}

with $R_c$ a constant that depends on the characteristics (e.g. geometry) of the TL, and is hence called the characteristic impedance. Although it has the units of resistance ($\Omega$) it is not reactive; there is no energy dissipation. The characteristic impedance can be interpreted as a scale for (1) the current to voltage amplitude and (2) the dissipation of energy to the load or generator (see (REF)). In further sections, $R_c$ is assumed to be finite and constant.

\subsection{Reflections and amplitude dampening}
In section (REF) the wave behavior of the voltage and current is discovered. To this end, the analysis of voltage waves at the boundaries for specific values of the generator resistance $R_g$ and load resistance $R_L$ is discussed in this section. \\

Our first question concerns the entrance of the bit to the TL: the simulation starts at $t=0$ and the generator $\hat{e}_g$ produces a bit with amplitude $V_0$. Before the bit enters the TL, it meets the generator resistance $R_g$ and thus energy will be dissipated i.e. the voltage amplitude is dampened. This can be clarified by expressing Kirchoff's Voltage Law at the origin ($z=0$) of the TL;

\begin{align}
&\hat{e}_g(t) = (R_g + R_c)\hat{i}(0, t) \\
&\Rightarrow \hat{i}(0, t) = \frac{1}{R_g+R_c}\hat{e}_g(t)= \frac{1}{R_c}(\hat{v}^{+}(0, t) - \hat{v}^{-}(0, t)) \\
&\Rightarrow \hat{v}(0, t) = \hat{v}^{+}(0, t) =\kappa\hat{e}_g(0, t).
\end{align}
The amplitude $V_0$ of generated bit is reduced by a factor $\kappa = \frac{R_c}{R_g + R_c}$. Here, the role of the characteristic impedance becomes clear; at the beginning of the TL, the bit 'sees' it as a resistance, but does not dissipate energy to it. Therefore the name input impedance. More interestingly, when $R_g = 0$, $\kappa = 1$ and the bit amplitude remains unchanged; the bit freely enters the TL since it does not meet resistance Figure (REF).














