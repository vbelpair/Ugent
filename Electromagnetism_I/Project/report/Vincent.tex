\subsubsection{Numerical implementation}

The update functions are given as:
\begin{align}
    \tilde{I}^{m+\frac{1}{2}}_{n+\frac{1}{2}} = \tilde{I}^{m-\frac{1}{2}}_{n+\frac{1}{2}} + \alpha\left(V^{m}_{n} - V^{m}_{n+1}\right) \label{update}, \quad\quad
    V^{m+1}_n = V^{m}_{n} + \alpha\left(\tilde{I}^{m+\frac{1}{2}}_{n-\frac{1}{2}}-\tilde{I}^{m+\frac{1}{2}}_{n+\frac{1}{2}}\right),  
    \end{align}
    
where
\begin{equation}
\alpha \triangleq \frac{v\Delta t}{\Delta z}
    \label{alpha}
\end{equation}

is the dimensionless Courant factor and
\begin{equation}
    \tilde{I}^{m+\frac{1}{2}}_{n+\frac{1}{2}} = I^{m+\frac{1}{2}}_{n+\frac{1}{2}}R_c
    \label{Itil}
\end{equation}
is the rescaled current.

At the boundaries the update function for $V$ takes another form.
\begin{itemize}
    \item At $z = 0$\\
    The voltage update function is given as:
    \begin{equation}
        V^{m+1}_{0} = V^{m}_{0} + \frac{2\Delta t}{C\Delta z}\left(I^{m+\frac{1}{2}}_{g} - I^{m+\frac{1}{2}}_{\frac{1}{2}}\right),
        \label{BC1}
    \end{equation}
    with 
    \begin{equation}
        I^{m+\frac{1}{2}}_{g} = \frac{E^{m+\frac{1}{2}}_{g}}{R_{g}}-\frac{V^{m}_{0}+V^{m+1}_{0}}{2R_g}.
        \label{Ig}
    \end{equation}
    Substituting (\ref{Ig}) in (\ref{BC1}) and using (\ref{alpha}), the two relations $v=\frac{1}{\sqrt{LC}}$ and $R_c=\sqrt{\frac{L}{C}}$ yields, 
    after some rearrangements:
    \begin{equation}
        V^{m+1}_{0} = K_{1}V^{m}_{0} + \kappa_{1}\left(E^{m+\frac{1}{2}}_{g}\frac{R_c}{R_{g}} - \tilde{I}^{m+\frac{1}{2}}_{\frac{1}{2}}\right)
    \end{equation}
    where
    \begin{align}
        K_{1} & = \frac{R_{g}-\alpha R_{c}}{R_{g}+\alpha R_{c}},\\
        \kappa_{1} & = \frac{2\alpha R_{g}}{R_{g}+\alpha R_{c}},
    \end{align}
    are two dimensionless constants named respectively the reflection coefficient and voltage spliiter coefficient.
    \item At $z = d$\\
    
    \begin{center}
    \includegraphics[scale=0.25]{BC2}
    \end{center}

    The voltage update function becomes:
    \begin{equation}
        V^{m+1}_{N} = V^{m}_{N} + \frac{2\Delta t}{C\Delta z}\left(I^{m+\frac{1}{2}}_{N-\frac{1}{2}} - I^{m+\frac{1}{2}}_{L}\right)
        \label{BC2}
    \end{equation}
    Kirchoff's voltage law in discretized form states that
    \begin{align}
        I^{m+\frac{1}{2}}_{L} & = \frac{V^{m+\frac{1}{2}}_{N}}{R_{L}}\\
        & = \frac{V^{m}_{N}+V^{m+1}_{N}}{2R_{L}}
        \label{IL}
    \end{align}
    Subsituting (\ref{IL}) in (\ref{BC2}) and using the same relations as for $z=0$ yields, 
    after some rearrangements:
    \begin{equation}
        V^{m+1}_{N} = K_{2}V^{m}_N + \kappa_{2}\tilde{I}^{m+\frac{1}{2}}_{N-\frac{1}{2}},
    \end{equation}
    where
    \begin{align}
        K_{2} & = \frac{R_{L}-\alpha R_{c}}{R_{L}+\alpha R_{c}},\\
        \kappa_{2} & = \frac{2\alpha R_{L}}{R_{L}+\alpha R_{c}},
    \end{align}
    are two dimensionless constants with the same names as the constants at $z=0$.

\end{itemize}

\subsection{Resistive and capacitive load}

\subsubsection{Numerical implementation}

Now if there would be a capacitor added to the load the following situation will occur
    
\begin{center}
\includegraphics[scale=0.35]{BC2_cap}
\end{center}

Kirchoff's current law states that
\begin{equation}
    I^{m+\frac{1}{2}}_{L} = I^{m+\frac{1}{2}}_{R} + I^{m+\frac{1}{2}}_{C}.
    \label{IL2}
\end{equation}
Using Kirchoff's voltage law at the resistor gives
\begin{align}
    I^{m+\frac{1}{2}}_{R} & = \frac{V^{m+\frac{1}{2}}_{N}}{R_{L}}\\
    & = \frac{V^{m}_{N}+V^{m+1}_{N}}{2R_{L}}
    \label{IR}
\end{align}
The relation between the current and the voltage at the capitor is given as
\begin{equation}
    \hat{i} = -C\frac{\partial \hat{v}}{\partial t}.
\end{equation}
For a first order FDM this turns into
\begin{equation}
    I^{m+\frac{1}{2}}_{R} = -C_{L}\frac{V^{m+1}_{N} - V^{m}_{N}}{\Delta t}.
    \label{IC}
\end{equation}
Substituting (\ref{IR}) and (\ref{IC}) into (\ref{IL2}) gives
\begin{equation}
    I^{m+\frac{1}{2}}_{L} = \left(\frac{1}{2R_{L}}-\frac{C_{L}}{\Delta t}\right)V^{m+1}_N + \left(\frac{1}{2R_{L}}+\frac{C_{L}}{\Delta t}\right)V^{m}_N
\end{equation}
and can be rewritten as
\begin{equation}
    \tilde{I}^{m+\frac{1}{2}}_{L} = \frac{R_{c}}{2Z_{1}}V^{m+1}_N + \frac{R_{c}}{2Z_{2}}V^{m}_N
    \label{IL2_final}
\end{equation}
with
\begin{align}
    Z_{1} &= \left(\frac{1}{R_{L}}-2\frac{C_{L}}{\Delta t}\right)^{-1} = \frac{R_{L}\Delta t}{\Delta t - 2 C_{L}R_{L}} \nonumber\\
    Z_{2} &= \left(\frac{1}{R_{L}}+2\frac{C_{L}}{\Delta t}\right)^{-1} = \frac{R_{L}\Delta t}{\Delta t + 2 C_{L}R_{L}} \label{eq:Z}
\end{align}
Substituting (\ref{IL2_final}) into the adjusted voltage update function at $z=d$ yields
\begin{equation}
    V^{m+1}_{N} = K'_{2}V^{m}_N + \kappa'_{2}\tilde{I}^{m+\frac{1}{2}}_{N-\frac{1}{2}},
\end{equation}
where
\begin{align}
    K'_{2} & = \frac{Z_{2}}{Z_{1}}\frac{Z_{2}-\alpha R_{c}}{Z_{1}+\alpha R_{c}},\\
    \kappa'_{2} & = \frac{2\alpha Z_{1}}{Z_{1}+\alpha R_{c}}.
\end{align}

\subsubsection{Influence of C}

First thing to notice is that when $C_{L} = 0$ then $K'_{2} = K_{2}$ and $\kappa'_{2} = \kappa_{2}$. This is a good
sign since it stays consistent with the previous update equation.\\
Secondly in relations \ref{eq:Z} there is a time step dependecy, hence the two dimensionless
coefficients in the boundary update function at $z=d$ are also dependent on the implemented time step. Let's consider a
well chosen time step and just look at the effect of changing $C_{L}$.
\begin{itemize}
    \item $C_{L} \rightarrow 0$ :
        $K'_{2} \rightarrow K_{2}$ and $\kappa'_{2} \rightarrow \kappa_{2}$, initial situation
    \item $C_{L} \rightarrow \infty$ :
        $ K'_{2} \rightarrow 1$ and $\kappa'_{2} \rightarrow 0$, fully reflected wave
    \item $C_{L} \rightarrow \frac{\Delta t}{2R_{L}}$ :
        $K'_{2} \rightarrow 0$ and $\kappa'_{2} \rightarrow 2\alpha$, would expect fully absorbed wave but get numerical problems
\end{itemize}

