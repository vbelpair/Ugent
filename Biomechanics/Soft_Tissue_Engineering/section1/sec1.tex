\documentclass[../conclusions.tex]{subfiles}

\begin{document}
    
\section{A first estimate of aneurysm rupture risk}

\subsection{Estimating tissue strenght based on a uniaxial tensile test}

The tissue strenght is here defined as the ultimate stress value obtained in a uniaxial tensile test, before damage occurs. Since uniaxial 
tensile tests were performed on circumferentially ($P^{ult}_{\theta\theta}$) as well as axially ($P^{ult}_{zz}$) oriented samples, we can define the strength 
in the circumferential as well as in the axial direction.  

\includegraphics[scale=0.5]{tissue_example.png}

The reaction forces were plotted against the tensile extension.

\begingroup
\begin{center}
    \setlength{\tabcolsep}{0.5cm}
    \renewcommand{\arraystretch}{1.5}
    \begin{tabular}{cc}
        \includegraphics[scale=0.3]{uniaxial_hourglass_circ_dataA1_extension_vs_load.png} &
        \includegraphics[scale=0.3]{uniaxial_hourglass_circ_dataB1_extension_vs_load.png} \\
        \includegraphics[scale=0.3]{uniaxial_hourglass_circ_dataC1_extension_vs_load.png} &
        \includegraphics[scale=0.3]{uniaxial_hourglass_circ_dataD1_extension_vs_load.png} \\
    \end{tabular} 
\end{center}
\endgroup

\end{document}